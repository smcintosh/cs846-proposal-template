\documentclass[conference]{IEEEtran}
\IEEEoverridecommandlockouts
% The preceding line is only needed to identify funding in the first footnote. If that is unneeded, please comment it out.
\usepackage{cite}
\usepackage{amsmath,amssymb,amsfonts}
\usepackage{algorithmic}
\usepackage{graphicx}
\usepackage{textcomp}
\usepackage{xcolor}
\usepackage{flushend}
\def\BibTeX{{\rm B\kern-.05em{\sc i\kern-.025em b}\kern-.08em
    T\kern-.1667em\lower.7ex\hbox{E}\kern-.125emX}}
\begin{document}

\title{Proposal Title}

\author{\IEEEauthorblockN{Shane McIntosh (replace with your name)}
\IEEEauthorblockA{\textit{Cheriton School of Computer Science} \\
\textit{University of Waterloo}\\
Waterloo, Canada\\
shane.mcintosh@uwaterloo.ca}
}

\maketitle

%\begin{abstract}
%\end{abstract}

%\begin{IEEEkeywords}
%component, formatting, style, styling, insert
%\end{IEEEkeywords}

\section{Motivation}

Briefly explain and motivate the problem.
Why is it an important problem?
Which stakeholders stand to (potentially) benefit from your research results and how so?

Adding citations here to a conference paper~\cite{gallaba2022icse} and a journal article~\cite{mcintosh2018tse} to illustrate how to use BibTeX (see \texttt{references.bib} for the bibliograpy entries).

\section{Data}

Explain where and/or how you will acquire the data that you need to conduct your study.
Consider breaking this down into subsections that explain the raw extraction (e.g., downloads of large archives of data) or collection process (e.g., writing crawler scripts or simulation experiment scripts), the cleaning process (i.e., how will you mitigate noise from the raw data), and the analysis process (how will you process the cleaned data to answer your research questions?).

\section{Project Milestones}

Break your project down into a list of milestones that can be used to make sure that your project is on schedule.
If you drift off schedule, the milestones should help you to recognize the need to pivot or reevaluate your plan.

\section{Expected Outcome}

Provide a list of contributions that you believe your work will make.
These may be conceptual contributions (e.g., results of empirical analyses, case studies), technical contributions (e.g., prototype tools, evaluated algorithms), or both.

\bibliographystyle{IEEEtranS}
\bibliography{references}

\end{document}
